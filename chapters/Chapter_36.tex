\chapter{Gestation Periods}
\label{cha:Lush_Chapter_36}
\index{Gestation periods|(}

A certain length of gestation is characteristic of each species; but,
like other characteristics, the lengths of individual gestation periods
vary. The actual union of the ovum and spermatozoon may be several
hours after the successful service. The gestation period is measured as
the number of days between the day of service and the day of birth of
the young, since the exact hour of fertilization will not be known.

In most of the animal kingdom except the mammals the fertilized
eggs hatch outside the body of the mother. There are exceptions to this
general rule. For example, there are some snakes and some fish in which
hatching takes place inside the body of the mother and others in which
it takes place outside, but in either case the young hatch in a similar
stage of development. Their embryology is more like that of birds than
like that of mammals. Even among the mammals a few of the most
primitive --- for example, the duck-bill of Australia --- lay eggs which
hatch outside their bodies in somewhat the same way as do those of
birds. In many of the primitive mammals (the ``marsupials''), the eggs
hatch within the body of the mother, but development proceeds only a
short way before the young are born at a very immature stage. After
they leave the mother's body, they are nourished and carried by the
mother in a pouch. The only North American representative of these
mammals is the common opossum.

In the higher mammals (``Placentalia''), the ovum not only starts its
development within the body of the mother but develops a special
mechanism, the placenta, by which its blood vessels come into close contact
but not actual union with those of the mother. Through the placenta
food and oxygen from the mother diffuse to the embryo, and the
carbon dioxide and other waste products from the embryo diffuse into
the blood of the mother. The placenta enables the embryo to develop
far before birth, while it is still sheltered against an environment hostile
in many respects. Because of the better care the mother can provide,
no such enormous number of fertilized eggs need be started on the road
to development in order that a few shall reach maturity safely as is the
case with most lower animals, e.g., the thousands or even millions of
eggs spawned per female among fishes.
\noclub[3]

All the higher mammals share the evolutionary advantages of the
placental mechanism, but they differ widely in the degree of maturity
which their young attain before they are born. For example, the young
of rabbits are born without their hair, with their eyes closed, and are
fully dependent on their mother for many days. At the other extreme,
although both are rodents, the young guinea pig is born fully equipped
with hair, with its eyes open, is able to eat lettuce and cabbage before it
is a day old and under favorable circumstances can even survive without
a foster mother if its own mother dies at its birth. Most farm animals
are intermediate in this respect.

\section*{CAUSES OF VARIATION IN GESTATION LENGTH}

Within the species there are breed differences which are usually
slight but can be detected when large numbers of each breed are compared
(see \textit{Jour. Ani. Sci.} 2:50--52 and 4:13--14 for examples).
Within the breed there are no doubt individual differences, although
it is rare that one female is so different from the average of the
breed and has enough gestations that one could be certain that she
really differs from the breed to which she belongs. A common cause of
variation is disease, such as abortion. Environmental conditions not
usually classified as disease may sometimes bring about a shortening or
lengthening of the gestation period. The season of the year in which
pregnancy begins has a slight effect on the length of the gestation period
in some cases. There is some evidence to indicate that nutritional conditions,
even when not severe enough to be actually pathological, may
affect the length of gestation. Within a species large litters are usually
carried for a shorter time than small litters. Various workers have at
times detected an influence of the sex of the offspring upon the length
of time it is carried, but the evidence is contradictory. This is probably
never a major cause of variation. For example, Uppenborn reports
from a study of 5,600 cases that stallion foals are carried an average of
$1.6 \pm .2$ days longer than mare foals --- a real difference, but not a practically
important one. On the same subject Mauch reports an average of
$1.7 \pm .5$. The size of the offspring may influence the length of time it
is carried. First gestations are often a little shorter than later ones.

\section*{``NORMAL'' AND ``ABNORMAL'' GESTATION PERIODS}

There is no entirely satisfactory test to determine whether a particular
gestation is normal in length or otherwise. This problem comes up
in a practical way most frequently in connection with cattle when the
question is raised as to whether a given gestation terminated normally
or in an abortion. Abortion may occur at all stages from the time when
the embryo is so small that it escapes detection, to the stage of a gestation
period practically normal in length. The "average length of gestation"
is a statistical composite picture of many individual cases, no one
of which may perhaps have been of exactly the average length. Some
variation is normal and is just as characteristic of the species as is variation
in other traits or organs.

As a practical guide to normal and abnormal gestations,
Tables~\ref{tbl:Lush_Table_21a} and
\ref{tbl:Lush_Table_21b}\footnote{Typesetter's note: This information appears in a single grouping,
Table 22, in the original text. The numbering of subsequent tables will be
off by 1 because of this.} show the standard deviation of gestation
lengths for the farm animals. Approximately two-thirds of all gestation
periods may be expected to differ from the average by less than the
standard deviation. A useful rule is to suspect of being abnormal any
gestation differing from the species average by more than the standard
deviation. Where the breed or herd average is known and is based on
reliable numbers, deviations should be figured from it instead of the
species average, of course. Any gestation differing from the average by
more than twice the standard deviation is likely to be abnormal and calls
for examination as to whether there may not be some diseased condition
which needs attention, or whether perhaps there may not have been some
mistake in the breeding date.

\begin{table}[t]
	\centering
	\caption{\textsc{Gestation Periods in Common Mammals. A. Summarized from published actual counts various breeds and places.}}
	\label{tbl:Lush_Table_21a}
	\begin{tabular}{L{3.5cm}|C{2.5cm}|C{2.5cm}|L{3.5cm}}
	\hline
	\hline
	~						& No. of Gestations	& ~					& Standard Deviation	\\
	~						& Included			& Average Length	& of Individual			\\
	Kind of Animal			& in Average		& of Gestation		& Gestation Lengths		\\
	\hline
	Horses					& 28,456			& 335.9				& About 10 or 11 days	\\
	Ass 					& 14				& 366.9				& 12 days				\\
	Mares bred to jacks		& 2,338				& 350				& About 19 days			\\
	Cattle					& 27,810			& 282.1				& About 5 days			\\
	Swine					& 6,535				& 114.3				& 2.2 days				\\
	Sheep					& 4,417				& 149.1				& 2.4 days				\\
	Goats					& 6,761				& 150.8				& 3.3 days				\\
	Dog						& 147				& 61.3				& 3.1 days				\\
	Rabbit					& 1,540				& 31.4				& 1.1 days				\\
	Silver fox				& 797				& 52.2				& 0.9 days				\\
	\hline																				
	\end{tabular}
\end{table}

\afterpage{
\begin{longtable}{L{3.5cm}L{8.5cm}}
	\caption{\textsc{Gestation Periods in Common Mammals. B. Quoted from various books.}} \\
	\label{tbl:Lush_Table_21b} \\
	\hline
	\hline
	Kind of Animal		& Average Length of Gestation Periods			\\
	\hline
	\endfirsthead
	\caption{\textsc{Gestation Periods in Common Mammals. B. Quoted from various books.} (\textit{Continued})} \\
	\hline
	\hline
	Kind of Animal		& Average Length of Gestation Periods			\\
	\hline
	\endhead
	\multicolumn{2}{L{12cm}}{\textit{Farm animals:}}						\\
	Jennet				& About 12 months. Quite variable				\\
	Mare				& About 11 months. Quite variable				\\
	Cow					& 280 to 285 days. 283 is most frequently stated	\\
	Ewe					& 147 to 150 days								\\
	Doe (goat)			& 149 to 154 days								\\
	Sow					& 112 to 114 days								\\
	Bitch				& 58 to 67 days, usually about 63				\\
	Cat					& 60 to 64 days. Some state 50 days				\\
	\multicolumn{2}{L{12cm}}{\textit{Laboratory animals:}}				\\
	Rabbit				& 31 days										\\	
	Chinchilla			& 111 days										\\
	Hamster				& 15 days										\\
	Guinea pig			& 69 days										\\
	Mouse				& 21 days										\\
	Rat					& 21 days										\\
	\multicolumn{2}{L{12cm}}{\textit{Other animals, about which there is less certainty$^*$:}}	\\		
	Bear 				& 6 months. Recent evidence indicates			\\
	~					& that this is too short						\\
	Beaver 				& 4 months. Another writer states 65 days		\\
	Buffalo 			& 10-12 months. (315 days with $\sigma = 5.5$	\\
	~					& days in the Asiatic buffalo)					\\
	Camel				& 13 months										\\
	Dromedary			& 12 months										\\
	Elephant			& 20--24 months									\\
	Ferret				& 42 days										\\
	Fisher				& 352 days										\\
	Fitch				& 43 days										\\
	Fox					& 52 days										\\
	Giraffe				& 14 months										\\
	Lion				& 3\nicefrac{1}{2} months						\\
	Marten				& 267 days										\\
	Mink				& 51 days										\\
	Monkey				& 7 months										\\
	Muskrat				& 21 days										\\
	Nutria				& 140 days										\\
	Opossum				& 12\nicefrac{1}{2} days						\\
	\hline
	Otter				& 55 days										\\
	Prjewalsky's horse	& 356--359 days									\\
	Puma				& 79 days. One writer says 15 weeks				\\
	Raccoon				& 65 days										\\
	Reindeer			& 8 months										\\
	Seal				& 11--12 months									\\
	Skunk				& 40 days. Some state 63 days					\\
	Squirrel			& 28 days										\\
	Tiger				& 22 weeks										\\
	Walrus				& One year										\\
	Wolf				& 60 days to 63 days							\\
	Zebra				& Same as the horse								\\
	\hline
	\multicolumn{2}{L{12cm}}{$^*$For more details see: Breeding Data on Fur Bearing Animals, special}		\\
	\multicolumn{2}{L{12cm}}{circular dated June, 1933, from the Department of Veterinary Science,}			\\
	\multicolumn{2}{L{12cm}}{University of Wisconsin; or: Kenneth, J. H. 1943. Gestation; a table}			\\
	\multicolumn{2}{L{12cm}}{and bibliography. Edinburgh. Oliver and Boyd.}										\\		
\end{longtable}
}

The standard deviations in Table~\ref{tbl:Lush_Table_21a} were calculated
from actual data which may in some cases have included errors in breeding
dates or may have included as normal some gestation periods which really
were terminated by abortion at such an advanced stage that the offspring
was able to live. So far as they could be detected, all such cases
were excluded from these data, but it is unlikely that all were detected.
The practical effect of including a few errors of this kind is to make the
standard deviations a little larger and the means a little smaller than
they would otherwise be.
\index{Gestation periods|)}

\section*{PRACTICAL USES FOR KNOWING THE LENGTHS OF GESTATION PERIODS}

First of all the caretaker needs to know when to prepare for the
young by isolating the prospective dam and fixing things so that she can
take good care of her new-born. It is not safe to rely entirely upon the
breeding calendar in doing this, both because individual animals may
vary distinctly from the expected date and because, after all, that animal
actually may have been bred at some other date than the one
recorded for it.

Another practical use for the knowledge of gestation length is in
settling cases of disputed parentage where a female may have been
bred to two different males at different heat periods and the young is
born at a time which does not correspond exactly to either service date.
Often such cases cannot be satisfactorily settled, and the only honorable
thing to do is to regard the offspring as a grade or at best as a purebred
not eligible to registry.

Observing gestation lengths carefully may help some in disease control
by enabling the breeder to quarantine each female several days
before she is due to produce her young and by calling his attention
immediately to any which deviate enough from expectation that they
are likely to need veterinary attention.