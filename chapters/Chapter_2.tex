\chapter{Consequences of Domestication}
\label{cha:consequences-of-domestication}

The fundamental laws of heredity and the mechanics and physiology of reproduction are the same among domesticated animals as 
among their wild relatives. The change from the wild to the domesticated condition did not alter these laws nor create any 
new inheritance. The changes which domestication did bring about were an increased amount of inbreeding and outbreeding and 
assortive mating and the addition of artificial selection to the forces of natural selection. The changes in environment 
which accompanied domestication doubtless permitted many differences in heredity to show themselves more clearly than they 
could in the environment of wild animals, and thus to be more readily and accurately selected. For example, when feed is so 
scarce that no animal gets all it wants, hereditary differences in ability to fatten could hardly show themselves as clearly 
as they can in a well-managed feedlot. But there is no direct evidence to indicate that the changed environment directly 
\textit{created} any new genes or hereditary differences.

\index{Inbreeding|(}\textit{Increased inbreeding} happened because domesticated animals were more narrowly restricted (by tethering, herding, or 
fencing) to growing up, remaining, and reproducing in the same region in which they were born. Soon the animals around one 
community or village would all become related to each other. Most breeders, even among primitive peoples, intentionally 
avoided the very closest inbreeding; but often the pedigrees were known only for a generation or two, or only in the female 
line. Under such circumstances, attempts to avoid inbreeding merely made the inbreeding less intense so that more time was 
required to produce the same amount of fixation and uniformity in the stock. That gave more opportunity for the accompanying 
selection to discard undesired results of the inbreeding than if the inbreeding had been extremely close. Even when 
obstacles to exchange between tribes were greatest, some introduction of outside blood went on either by trade or by war. 
Doubtless these exchanges usually involved stock from only the closest neighboring tribes. Such animals, as a result of 
previous exchanges, would already be more closely related to those among which they were introduced than animals which might 
have been chosen at random from the whole species. There would have had to be a large amount of such interchange to prevent 
this inbreeding from bringing about a situation where almost every tribe or community would have its own type of each kind 
of animal. Doubtless the intensity of this inbreeding varied greatly from region to region according to the nomadic habits 
of the people, geographic barriers, or social customs which may have prevented extensive trade with neighboring tribes, and 
the extent to which their beliefs about breeding led them to try deliberately to prevent this inbreeding by taking special 
pains to get sires from unrelated or remotely related stocks. This extra inbreeding resulting from restricting the 
movements, and also from limiting the number of breeding males, may well have been one of the most potent forces leading to 
the production of diverse races among domestic animals. It does not seem possible now to measure its past importance 
accurately, since so little is known about actual breeding customs until very modern times. The geographic or physiologic 
isolation, which in nature divides so many species into small sub-groups between which crosses occur only at rare intervals, 
is the very same kind of a process in principle, but probably is rarely as extreme in nature as it is under domestication 
where man has added so many artificial barriers to the natural ones.

\index{Outbreeding|(}\textit{Increased outbreeding} has certainly resulted now and then from domestication. By the agency of man, breeding 
animals could be transported far beyond the area in which they were born or over which they could have wandered before they 
were domesticated. Thus they could be crossed on races more diverse than they would have encountered if they had remained in 
the wild state. Knights returning from the Crusades brought with them stallions from Arabia. Cattle were brought from 
Holland to England across water which would have been impassable to wild cattle. In a later day Merino sheep were taken from 
Spain to many lands, Angora goats came from Turkey to the United States and to South Africa, zebu cattle came from India to 
Brazil and to the Gulf Coast of the United States, and Shorthorn cattle went from England to the Argentine and Australia. 
Many other examples could be cited to show how this process went on in historic times. The diverse races of swine from which 
the Poland-China breed was formed could hardly by any conceivable circumstances have come together anywhere on earth without 
the intervention of man to transport them. 

Presumably this process has gone on more rapidly in the last three or four centuries of exploration and widespread trade
than it did formerly, but Phoenician traders were traveling the length of the Mediterranean and skirting the western shores 
of Europe as far as Britain nearly three thousand years ago and doubtless helped exchange some of the smaller animals at 
least. Invading armies or migrating peoples usually carried with them much livestock from their native lands; some of these 
were mingled with the breeding stock of the countries through which they passed. Two thousand years ago Hannibal and 
Hasdrubal took their armies and livestock, including such large and unwieldy animals as elephants, from Carthage along the 
northern coast of Africa to Spain and over such mountains as the Pyrenees and Alps almost to Rome itself. Also, Alexander 
the Great took large numbers of livestock with his armies on the road from Greece to India and had with him men who made 
careful notes about the strange livestock they saw in the new lands. In the early thirteenth century the armies of Genghis 
Khan and his sons, with their enormous reserves of cavalry horses, ranged all the way from eastern China to central Europe. 
The passing of that band of horses (with the inevitably large amount of straying and robbing) must have changed greatly the 
genetic composition of the local races in the regions through which they passed. With such cases known from definite 
history, it is only reasonable to suppose that similar exchanges by migrations or wars had been taking place almost since 
the beginnings of domestication. Here again, as in the case of geographic isolation, there must have been much more of such 
exchanges in some parts of the world than in others. 

A combination of moderate inbreeding alternating with occasional wide outbreeding is an effective plan for producing many 
distinct families which are moderately uniform within themselves. A population being thus bred is in a more favorable 
condition for selection to be effective than if matings within the group selected to be parents were entirely at random. In 
this way domestication made conditions more favorable for the formation of distinct races than exist among wild animals. 
\index{Inbreeding|)}
\index{Outbreeding|)}

\index{Selection|(}
\textit{Selection} means differences in reproductive rates within a population, whereby animals with some characteristics 
tend to have more offspring than animals without those characteristics. Thereby the genes of the favored animals tend to 
become more abundant in the population and those of the less favored animals less abundant. Artificial selection differs 
from natural selection only in the kind or degree of the characteristics which are thus favored. Also, in many cases 
artificial selection may be more intense, less of the decision being left to chance or to accidental circumstances than in 
the case of natural selection.

Natural selection did not wholly cease with domestication. More of the weak and sickly than of the strong and vigorous are 
still doomed to die before they reach breeding age. This will happen whether the breeder consciously aids in this selection 
or not. Indeed, some of it will happen in spite of the breeder's efforts if he tries to breed a type which is rather frail 
or susceptible to disease. Among domesticated animals natural selection is merely supplemented by man's selections. In 
making his decisions as to which animals should leave few and which should leave many offspring, man often strongly 
emphasizes characteristics which were of little worth in a state of nature. Other qualities valuable in the wild state 
became useless or nearly so when man began to protect his animals against their enemies, against cold and against 
starvation. Thus, man's selection may differ from natural selection both in \textit{intensity} and in \textit{direction}.

The practice of favoring for breeding purposes those animals which in their owner's opinion were the most desirable ones 
must have begun with domestication itself. The recognition that offspring tend to resemble their parents and other near 
relatives occurred at so early a cultural stage that proverbs embodying this idea are found in practically all languages, 
even those of extremely primitive people. Primitive man was doubtless quite shrewd enough to put this knowledge into 
practice on his domesticated animals. Castration is one of the most ancient of surgical practices. Medical literature 
traces it back at least as far as 700 \textsc{B.C.}, and in the Bible there is frequent reference to eunuchs in the time of 
Solomon or earlier. Eunuchs are mentioned in the Code of Hamurabi; ca. 2000 \textsc{B.C.} Castrated bullocks (balivarda), 
as distinct from bulls, are mentioned in the Puranas of the Hindus, which deal with the events of the Aryan migrations into 
northwestern India probably as long ago as 2400-1500 \textsc{B.C.} (Since the Puranas were not put into writing until much 
later, the words may have been changed during the interval). The general practice of castration must have intensified the 
selection which was practiced among males.

The Roman agricultural literature of about two thousands years ago\footnote{Harrison, Fairfax. 1917. \textit{Roman Farm 
Management}. 365 pp. New York: The Macmillan Company.} contains many bits of advice about the kinds of animals to select 
for different purposes. Much of this they had copied from still older writings, such as those of Mago the Carthaginian. It 
is certain that artificial selection has been practiced by man for thousands of years, although there seems to be no way of 
measuring the intensity with which it was practiced.

Another aspect of selection which was intensified by domestication was that the characteristics favored for one set of 
conditions might not be the same as those favored under other conditions. This, of course, was true in nature also, 
contrasting characteristics sometimes being favored for life on the open grasslands or for life in the forest, for 
mountain or lowland, for tropics or temperate climates, etc. In nature these would often --- perhaps usually --- be 
characteristic of wide areas with broad transitional zones between them. Under domestication one man might prefer a certain 
type of horse or cattle, and his next-door neighbor or the people of the very next village might prefer and select for a 
distinctly different type. Indeed, as agriculture became more complex the same man might keep two or more types of the same 
species, each being preferred for some special purpose. For instance, the ancient Egyptians had several breeds of dogs, and 
modern farmers keep different breeds of cattle for beef and for dairy purposes. So far as one breed or type is concerned, 
this is nothing but selection directed toward that particular ideal; but, from the standpoint of the whole species, this is 
assortive mating, which is a powerful tool for producing diversity within a species. This seems certain to have been more 
intensified under domestication than it was in nature.
\index{Selection|)}

\section*{SUMMARY}

Domestication merely intensified forces or processes which already existed in nature. Increased inbreeding alternating with 
wider outcrossing, more intense selection devoted toward a wider variety of goals, and mating like to like wherever one 
man or tribe was breeding the same species for two or more different goals, all had the net effect of tremendously speeding 
up the slow process of evolution as it occurs in nature, until remarkably large changes were made in animals under 
domestication during what was a very short period in terms of geologic time. That the changes thus brought about were at 
the maximum rate possible seems highly unlikely. If further changes are desired, it is probable that the possibilities in 
most directions are by no means exhausted and that intelligent use of these same processes can result in much faster 
progress than has been averaged during the long (in terms of human lifetimes) history of domestication.
\index{Domestication|)}