\chapter{Sex Ratios}
\label{cha:Lush_Chapter_37}
\index{Sex ratio|(}

The proportions of the two sexes are approximately equal in all the
vertebrates. There are other animals among which sex ratios are normally
very far from equality. For example, in some of the gall flies the
males may be as rare as 1 or 2 per cent of the population. There are
slight but consistent deviations from equality in the sex ratio even
among the higher animals. A summary of the usual percentage of males
among the total births for several species is shown in
Table~\ref{tbl:Lush_Table_22}.

\afterpage{
	\label{tbl:Lush_Table_22}
	\begin{longtable}{L{3cm}|C{2.5cm}|C{2.5cm}|L{3.5cm}}
		\caption{\textsc{Sex Ratios in Several Species of Animals}} \\
		\hline
		\hline
		Animal					& Percentage of Males Among All Births	& Approximate Number of Births Studied		& Author, Date, or Notes	\\
		\hline
		\endfirsthead
		\caption{\textsc{Sex Ratios in Several Species of Animals} (\textit{Continued})} \\
		\hline
		\hline
		Animal					& Percentage of Males Among All Births	& Approximate Number of Births Studied		& Author, Date, or Notes	\\
		\hline
		\endhead		
		\textit{Farm animals:}	&				&				& 							\\
		Horse					& 49.7			& 1,111,908		& D\"using					\\
		``						& 48.9			& 34,497		& Richter					\\
		``						& 49.3			& 11,261		& Uppenborn					\\
		``						& 49.7			& 62,002		& Schlechter				\\
		``						& 49.1			& 4,109			& Lauprecht, 1932			\\
		``						& 49.9			& 25,560		& Darwin					\\
		~						&				&				& 							\\
		Mule					& 44.3			& 1,416			& Craft, 1933				\\
		~						&				&				& 							\\
		Cattle					& 50.5			& 3,559			& Gowen and Pearl			\\
		`` (Dairy)				& 49.4			& 13,000		& Roberts, 1930, and Roberts and Yapp, 1928			\\
		``						& 51.5			& 124,000		& Johansson, 1932			\\
		``						& 49.9			& 20,579		& Engeler, 1933				\\
		``						& 52.2			& 11,450		& Ward, 1941				\\
		~						&				&				& 							\\
		Sheep					& 49.5			& 91,640		& Chapman and Lush, 1932	\\
		``						& 49.0			& 127,587		& Henning, 1939				\\
		~						&				&				& 							\\
		Goat (Angoras)			& 50.1			& 3,000			& Lush, \textit{et al.}, 1930 \\
		~						&				&				& 							\\
		Swine					& 52.3			& 23,000		& McPhee, \textit{et al.}, 1932 \\
		``						& 50.6			& 48,000		& Krallinger, 1930			\\
		``						& 51.1			& 3,639			& Hetzer, \textit{et al.}, 1933 \\
		``						& 50.1			& 5,373			& Kozelhua, 1940 			\\
		~						&				&				& 							\\
		Dog						& 51.5			& 50,000		& Whitney, 1927				\\
		``						& 52.8			& 159,304		& Winzenburger, 1936		\\
		``						& 52.4			& 324,323		& Druckseis, 1935			\\
		~						&				&				& 							\\
		Cat						& 55.0			& 653			& Jones, 1922 (embryos only) \\
		~						&				&				& 							\\
		Chicken					& 49.4			& 102,143		& Juli, 1940				\\
		\hline
		Chicken					& 50.8			& 23,273		& At eight weeks. Hays, 1941 \\
		\hline
		\textit{Other mammals:}	&				&				& 							\\
		Man 					& 50.7 to 51.7	& $\cdots$		& Various authors			\\
		``						& 51.4			& $\cdots$		& At birth. Crew, 1937		\\
		``						& 52.4			& $\cdots$		& Stillbirths. Crew, 1937	\\
		``						&				&				& 							\\
		Rat						& 51.2			& $\cdots$		& Cu\'{e}not				\\
		``						&				&				& 							\\
		Mouse					& 50.0 to 54.1	& $\cdots$		& $\cdots$					\\
		``						&				&				& 							\\
		Rabbit					& 51.1			& $\cdots$		& $\cdots$					\\
		``						&				&				& 							\\
		Guinea pig				& 50.8			& $\cdots$		& Ibsen						\\
		``						& 49.4			& 2,014			& Schott, 1930				\\
		\hline
	\end{longtable}
}

These are the sex ratios among those actually born. Sometimes these
are called secondary ratios to distinguish them from the primary ratios
which exist at the time of conception. The secondary sex ratio can differ
from the primary one if there are differences in the prenatal mortality
of the two sexes. Such differences are known to exist in several species.
Apparently it is the general rule among the mammals that the prenatal
and also the immediate postnatal mortality is higher among males than
among females. It is usually impossible to determine the primary sex
ratio directly. In practically all writings on the subject the sex ratio
stated is that at birth unless otherwise specified. Some writers even distinguish
a tertiary sex ratio, which would be the ratio of the sexes existing
at maturity or at some other age-perhaps at weaning time for the
farm animals. The tertiary sex ratio is so influenced by postnatal conditions
of management that it is rarely used.

Sex ratios are usually expressed as the number of males per hun dred
females or as the percentage of male births among all births
studied. The first method has the disadvantage that anything producing
a certain effect on one sex would magnify the proportion more
than if it produced the same effect on the other sex. For example, if anything
occurred to destroy one-fourth of the males and the sexes had
really been present in exactly equal number at the start, the sex ratio
would be stated as 75 males to 100 females. On the other hand, if something
had destroyed one-fourth of the females, the sex ratio would be
stated as 133 males to 100 females. In the latter case the deviation would
at first glance appear larger, although the amount of change is really
the same. When expressed as percentages of the total births, such
changes would appear of equal size, regardless of the sex in which they
occurred.
\noclub

\section*{CAUSES OF VARIATION IN SEX RATIOS}

The deviations of sex ratios from exact equality are small, but some
of them are based on too large numbers to be accidental. Hence they
have aroused the interest of biologists, out of all proportion to the
economic importance of such small deviations. There is an enormous
literature on the subject of sex ratios and the causes of their deviations
from exact equality. Crew's book\footnote{Crew, F. A. E., 1927,
\textit{The genetics of sexuality in animals}, Cambridge: The University
Press.} will serve as an introduction to that subject, but one has only
to look under ``sex ratios'' in the indexes of \textit{Biological
Abstracts} or of the \textit{Experiment Station Record} to note the large
amount written on that subject.

The usual cause of deviations from equality where the numbers arc
small is chance. Among a group of 32 cows having 5 calves each, the
most probable single result is that one cow will have only heifer calves
and one only bull calves, five cows will each have one daughter and four
sons and another five will each have four daughters and one son. The
remaining 20 will each have two sons and three daughters or two
daughters and three sons. It is to be expected that extreme deviations
will occur just by chance. Those are sometimes impressive to persons
who do not have firsthand familiarity with the wide variation which
chance can produce in small samples.

The most probable causes for the slight but real differences found
between sex ratios and exact equality are differential mortality of the
embryos of the two sexes and differences in the motility or longevity of
the two kinds of spermatozoa. The latter would lead to the initial production
of more embryos of one sex, even though the two kinds of gametes
were produced in equal numbers. Among the mammals mortality is
a little higher among the males than among the females at all ages, but
the reverse is true in birds. Sex-linked lethal genes are well known in
Drosophila and lead to abnormal sex ratios. Dr. King's partial
success\footnote{\textit{Journal of Experimental Zoology}, 27:1--35.}
in selecting two strains of rats for a high and a low sex ratio may have
been based on such lethals or on lethals which affected only one sex. In
plants there is good evidence of differences in the rates at which various
kinds of pollen tu bes grow down through the maternal tissue to reach
the ovules. It is doubtful that anything as extreme as this is important
in the higher animals; yet there might be enough of it to explain the
slight deviations from exact equality which actually exist.

Various investigators have found that differences in sex ratios were
sometimes associated with such things as race, season of the year, year-to-
year differences, excessive sexual activity, etc.; but none of these are
large enough to be economically important, although they do challenge
the investigator to explain them. Species crosses sometimes result in an
unequal sex ratio, but such crosses are too rare to be important in
practical animal breeding. The cause in these cases is the disturbed balance
between two different sets of sex-determining genes. For example, Cole
and Painter found 332 males but only 10 females among hybrids
between the pigeon and the ring dove. Sex was not determined on 418
others which died in the first week of incubation or on 89 others which
died later. Presumably high mortality among the females was the main
cause of the extreme sex ratio. In pigeons the females are heterogametic.

\section*{THE POSSIBILITY OF SEX CONTROL}
\index{Sex control|(}

It seems unlikely at present that the breeder will ever be able to
control the sex of the young produced. Apparently such control would
require either some treatment which would destroy the fertilized eggs
which are of one sex but leave unharmed those of the other sex, or it
would require some sort of treatment before fertilization which would
separate the two kinds of spermatozoa or would kill one kind and leave
the other unharmed. The first method, even if such a highly selective
treatment were discovered, would be impractical in the case of multiple
births because it would merely reduce the fertility by eliminating those
of the undesired sex without much if any compensating increase in
those of the desired sex. The second method conceivably might be of
some practical use but would require some chemical or some method of
treatment so delicately balanced that it would destroy spermatozoa of
one kind without harming those of the other kind or would separate
one kind from the other without harming the fertilizing ability of the
desired kind. The improbability of there being such a treatment, or of
finding it even if it does exist, becomes evident when it is considered
that in the farm animals the two kinds of spermatozoa would be exactly
alike in well over 90 per cent of the material they contain.

There have been an enormous number of theories of sex control
and sex determination. Drelincourt, writing in the seventeenth century,
named 276 ``false theories'' of sex determination. It is only fair to add
that his own theory was the two hundred seventy-seventh false one!
Geddes and Thomson, writing in 1901, estimated that the number of
published theories of this kind had doubled since Drelincourt's time.
Many of these theories are so vague or mystical that it is not possible to
test them experimentally. Such are the theories which invoke differences\
in ``potency,'' ``mental states,'' and the like. Many of those which
have a physical basis are susceptible to experimental tests. All which
have been tested so far can be explained on the chromosome theory of
sex determination. Even the cases of sex reversal have only shown that
certain environmental circumstances may at times be powerful enough
to turn the balance in the other direction from that in which the chromosome
mechanism would normally have thrown it. The same thing
has been done in dioecious species of plants, such as hemp, where sex
reversal can be accomplished by the proper combination of environmental
circumstances.

Any theory of sex determination, no matter how absurd, will be right in
about half of the cases, just as a matter of chance. The laws of chance permit
small samples to deviate widely from the expected average.\footnote{\textit{Human
Biology} 9:99--103.} Hence it is to be expected that even the most absurd theories will
sometimes seem to fit rather well a sample consisting of a few cases. Not
many years ago a man who thought he had discovered such a theory of
sex determination advertised widely in livestock magazines that if
breeders would pay him 50 cents per head he would tell them how to
get calves of the desired sex. He guaranteed to refund the money whenever
the results were not as he had predicted. Now he could be expected
to be right half of the time, just as a matter of chance. If he had
obtained enough business, this would have been a profitable undertaking
for him, because at the most he would only have to return half the
money he took in. Moreover, many probably would not ask for their
money back or would lose their receipts, etc. In this particular case the
man making these advertisements seemed to be sincere in his belief that
he had discovered some natural law and was only trying to profit from
his discovery as an inventor would from a patent. No doubt he would
have been indignant if he had been accused of fraud. The critical test in
such cases is to predict what the future sex will be in enough cases to
give the laws of chance an opportunity to work out. If the predictions
are not correct in significantly more than half the cases, there is no real
evidence to support the supposed method of sex control. The subject
has enough appeal to popular interest that any supposed new discovery
bearing on it is almost sure to get headlines and wide
publicity.\footnote{\textit{Journal of Heredity}, 24:264--74.}
\index{Sex control|)}
\index{Sex ratio|)}