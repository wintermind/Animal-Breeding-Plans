\chapter[Beginnings of Pedigree Breeding and Registry Societies]{The Beginnings of Pedigree Breeding and the Formation of Breed Registry Societies}
\label{cha:breed-registry-societies}
\index{Origin of breeds|(}
\index{Pedigree breeding, beginnings of|(}

\begin{quote}
``The virtues of their fathers live on in bulls and in stallions.'' --- Horace
\end{quote}

\begin{quote}
``Who would grow spirited stallions for the Olympic prizes or strong bulls for the plow, let him choose carefully 
the females who will be their dams.'' --- Virgil
\end{quote}

Emphasis on ancestry in human genealogies is older than history, although human pedigrees may have been used more for 
social purposes, such as to determine the inheritance of property or of rank in a caste system of society, than because of 
definite beliefs about the inheritance of physical and mental qualities. Often these pedigrees recorded only the male or 
only the female line of descent. The genealogies in the early chapters of Genesis are examples of this. 

Pure breeding is also an ancient idea as applied to man, not only in those peoples which had a pronounced caste system of 
society but also in many others where a tribe was warring on neighboring tribes, or a conquering race was trying to live 
with but keep itself ``pure'' from a conquered or slave race. Also, many tribes or races with a simpler social structure 
cherished myths about their racial origins which implied that they alone were the chosen people descended from the sun or 
the moon or some other deity, while other peoples were of inferior or mixed descent. In most (but not all) human societies 
there has been a heavy social prejudice against the ``half-breed,'' which in general has meant that the half-breed and its 
descendants must come up to higher standards of individual excellence than the average ``purebred'' before they would have 
an equal chance to contribute to the inheritance of the future race. 

The Arabs in their horse breeding more than a thousand years ago were memorizing the genealogies of their horses, but we 
have no detailed knowledge of how these genealogies were used --- if at all --- to guide them in making the matings.\index{Origin of breeds} 
Probably, like the modern Arabs, they traced the pedigrees only in the female line and used the family name only as an aid 
to selection,\footnote{Nurettin, Aral, and Selahattin, E. 1935. \textit{Der heutige Stand der Pferdezucht in 
Arabien.} Zeit. f. Z\"{u}chtung. (Reihe B), 33:13-38.} also taking some care to avoid close inbreeding. The Romans of the 
time of Varro and Cato made many comments about the kinds and types of animals which should be selected for breeding 
purposes but apparently made no attempt to memorize or record long pedigrees for their livestock. Varro's comments on the 
importance of judging the breeding worth of a sire by the quality of his get show that in a general way they were aware of 
the importance of the progeny test and the use of pedigrees, traced at least to the parents and grandparents, to help them 
to a more correct estimate of an animal's breeding worth.

``Throughout the Middle Ages the authority of the written word almost completely displaced firsthand observation and 
experiment in the search for truth.''\footnote{Mees, C. E. Kenneth. 1934. Scientific Thought and Social Reconstruction. 
\textit{Sigma Xi Quarterly}, 22(No. 1):17.} Largely because of this, knowledge of the mechanics and laws of heredity 
advanced but little. Most of the learning was preserved in the monasteries and what little is known about agriculture in 
the middle ages comes mostly from the account books, inventories, and fragmentary notes kept in connection with the farming 
operations of the monasteries. It was not until 1700 that enough was written and preserved to give us a connected account 
of agricultural practices.

The use of pedigrees in the modern manner began in rural England late in the eighteenth century, and the general formation 
of breed registry societies\index{Origin of breeds} began around the middle of the nineteenth century. Robert Bakewell is generally given credit 
for setting the pattern of modern animal breeding and is sometimes called the founder of animal breeding. Perhaps this is 
giving too much credit to one man, but at any rate pedigree breeding was established in his time, and his own outstanding 
success had more to do with making it popular than the efforts of any other one man did.

\index{Bakewell, Robert|(}Robert Bakewell was an English farmer or country gentleman who lived from 1725 to 1795.\footnote{A more complete account of 
Bakewell's work and of the conditions of animal breeding then is given on pages 176-189 of Lord Ernie's \textit{English 
Farming, Past and Present}. 1936.} We first hear of his agricultural efforts when he began to manage the estate at Dishley 
in 1760. He wrote little or nothing about himself. He was a good farmer in other things besides his stock breeding, having 
taken a prominent part in the introduction of turnips and other root crops into English agriculture. He was a good 
observer, a keen student of anatomy and probably a good judge of livestock. According to some accounts he even kept for 
future reference specimens of the bones or pickled joints of animals which he had bred and which he regarded as nearly 
ideal. He told so little about his operations that many of his contemporaries thought there was something mysterious about 
them. Some writers hint that this was done deliberately to avoid competition or censure. This latter point is made because 
an important element in his procedure was the deliberate and intense use of inbreeding. At that time there was even more 
prejudice against inbreeding than there is today, and many people thought it almost sacrilegious. Perhaps Bakewell thought 
it no use to invite criticism by proclaiming openly what he was doing. There is also more than a hint that he kept his 
operations secret because of certain extreme outbreeding he was practicing which, if known, might have injured the 
commercial reputation of his stock. Thus, there were rumors of a mysterious black ram used in his sheep breeding which 
visitors were never permitted to see and whose existence he would never admit.

Bakewell's own breeding work was with the old Longhorn cattle, Leicester sheep and Shire horses. He was so successful with 
these that his animals came into great demand as breeding stock. He inaugurated the practice of ram-letting. That is, he  
did not sell his best males out-right, but rented the use of them a year at a time. His annual auctions, or ram-lettings, 
attracted great attention and were a distinct financial success. He is said to have received as much as 1,200 guineas for 
one year's use of a ram. By this practice of ram-letting, the best sires came back to him each year and any whose progeny 
had proved them much better than the others could be kept for use in his own flocks or herds. There seems to be no record 
of how many times he took back for his own use a sire which originally he had thought not quite good enough for that, but 
no doubt such instances occurred. 

Bakewell's success attracted many imitators. From many parts of England ambitious stockmen went to Dishley to work with 
Bakewell and study his methods. Some of them stayed for as much as six months. Returning home they applied his methods to 
stock secured from him or to what they thought were the best of their own local animals. The details about these students 
and what they did are poorly known, but it is certain that the Collings who laid the foundations of the Shorthorn breed 
were in close touch with Bakewell and that men from Herefordshire were students with him. Enough of Bakewell's followers 
won distinct success that here and there all over England there soon began to be groups of animals closely related to each 
other and similar in type.These were the groups from which came the modern breeds, most of which were not formally 
organized as such until later. 

The \textit{principles} which Bakewell used included such things as: ``Like produces like or the likeness of some ancestor; 
\index{Inbreeding|(}inbreeding produces prepotency and refinement; breed the best to the best.'' His greatest contribution to breeding methods 
lay in his appreciation of the fact that produces like or the likeness of some ancestor; 
inbreeding was the most effective tool for producing refinement and fixing type. 
He was reluctant to make any \index{Outcross}\index{Outbreeding}outcrosses at all when his own stock seemed to him better than that of his neighbors. With his 
willingness to inbreed was coupled a good knowledge of anatomy and a keen interest in the subject of what types of animals 
were best suited to his agriculture and should be set up as goals. 
\index{Bakewell, Robert|)}

The economic setting of the times, of course, had much to do with the increased interest in breeding improved animals. The 
enclosures of the ``common lands'' had given the individual farmer opportunity to breed his own stock as he pleased and to 
reap the rewards of anything he might do to improve them or to build up the fertility of his own lands. The introduction of 
clovers and root crops to English agriculture had made more intensive animal husbandry possible and had supplied a store of 
roughages suitable for winter feeding. The times were ripe for commercial appreciation of animals which could utilize the 
crops of the new agriculture better than their contemporaries and which would produce a quality of product well suited to 
contemporary market demands. The warfare through the latter half of the eighteenth century, finally coming to a climax in 
the Napoleonic wars, often made prices high for farm products. Afterward, the industrial revolution and the steadily 
expanding urban population of Britain made a rising market for agricultural products, more particularly for those like meat 
which, in the days before refrigeration, could not be imported in the fresh state from the New World. When the improvement 
which Bakewell and his followers had made in their breeding stock began to be known in other lands, the export of breeding 
stock to those lands became a considerable source of income to British stockmen. Appreciation of the importance of this was 
a spur to further improvement in order to keep the foreign customers coming back for fresh breeding stock and had much to do 
with guiding the policies of breed registry societies.\index{Origin of breeds}
\index{Inbreeding|)}

\section*{BREED REGISTRY SOCIETIES}
\index{Breed associations|(}

As long as each breed was local the private records of each breeder were adequate for his own purposes. He usually knew at 
least the sires used by his fellow breeders and knew the integrity of those breeders well enough to have some idea of how 
much he could depend on their statements or records when purchasing breeding animals from them. But in time the number of 
breeders increased until many of them were utter strangers to each other, and the number of animal generations in the 
pedigrees increased until no man could remember all of the foundation animals far back in the pedigrees. To supply this 
knowledge and to prevent (as far as other breeders could) unscrupulous traders from exporting grades or common stock as 
purebreds, herdbooks were formed. The latter motive was very important\footnote{For example, many events in Bates' book
(1871) illustrate the incentive which the American demand for Shorthorns gave to the formation of the Coates herdbook.} 
and generally the herdbook was established soon after there began to be a considerable export demand.

\index{Herdbooks|(}
The first herdbook was ``An Introduction to the General Stud Book'' for the Thoroughbred horse and appeared in 1791. In it 
were recorded the pedigrees of the horses winning important races. It was aimed, therefore, at recording the pedigrees of 
performers rather than of all members of the pure breed. The Shorthorn herdbook, which first appeared in 1822, was the next 
one formed and may be taken as an example of the modern type of herdbook which aims at including the pedigrees of all 
animals of the pure breed. The Shorthorn herdbook, however, like the one for the Thoroughbred horse, would accept for entry 
outstanding individuals or performers which would be called high grades in the United States. Later herdbooks were largely 
modeled after the more successful of the early ones. An English Hereford herdbook was published in 1846 and a Polled Herd 
Book (for Aberdeen-Angus) in 1862. The first swine herdbook in the world was that of the American Berkshire Association, 
which appeared in 1876. The Berkshire Society in England was established first in 1883. The first herdbooks in the 
continental countries of Europe appeared at a later date than in Britain. Studbooks for horses were founded in France in 
1826, in Germany in 1827, and in Austria in 1847. The first cattle herdbook in France was established in 1855, the first 
German one in 1864, the first Dutch one in 1874 and the first Danish one in 1881. 

The Shorthorn herdbook was undertaken as a private venture by George Coates, who had been a Shorthorn breeder in a small 
way. A number of the Shorthorn breeders helped finance him and each was to receive a copy of the book. The other copies 
were to be his personal property to sell for whatever profit he could. He was already acquainted with many breeders, and 
from each of them he secured such information as he could about the animals which that man regarded as genuine Shorthorns. 
No doubt there was plenty of dispute about that. That is, some breeders would say that certain animals should be included 
in the records while others would think that those animals were not sufficiently desirable to be included as genuine 
Shorthorns. Coates was criticized, of course, for many of these decisions; and it was even charged\footnote{\textit{See} p. 
38 in Bates' \textit{History of Improved Shorthorn or Durham Cattle.}} (perhaps unjustly) that his favoritism to his 
personal friends went to the extent of printing for their cattle false pedigrees which would make them sell well to the 
American trade, then becoming important. Where Coates included animals which the majority of the breeders thought were not 
really pure, the breeders themselves could remedy the situation by having nothing to do with those animals or their 
descendants -- a course of action which is still open today and which is still used freely wherever falsification of 
pedigrees is suspected but evidence is not complete enough to justify canceling the registration. Wherever Coates omitted 
from the first volume of his herdbooks animals which the majority of breeders thought should have been included, such 
mistakes could be corrected by including these pedigrees in succeeding volumes of the herdbook -- a process no longer 
available wherever herdbooks are entirely closed to all but the offspring of registered parents.

Some of the very early breeders objected to furnishing pedigrees of the animals they sold, believing that they would thus 
give away valuable trade secrets. The demand for full information about pedigrees, however, finally prevailed over the
``trade-secret'' idea; and it became accepted as a matter of course that anyone selling breeding stock should furnish full 
identification of their immediate ancestors.

Doubtless many of the contemporary breeders felt that this herdbook of Coates was only a hobby of his which would disappear 
with his death; but, as the breed became more popular and the number of breeders increased and the number of generations to 
be remembered in the pedigrees grew larger, the difficulties which first prompted Coates to the formation of the herdbook 
became greater. Eventually every breeder admitted the necessity of the herdbook, in view of the customer demand for 
pedigrees, and depended upon it in his purchases and sales of breeding stock. When this stage was reached, those who owned 
the herdbooks had the power to charge exorbitant fees for registration and transfer or to use their influence to favor the 
business of certain breeders and to harm that of others. While there was rarely any widespread complaint of this kind, yet 
it generally seemed wiser or even necessary for the breeders to organize breed associations in order to manage the 
herdbook, conduct breed promotion, and attend to any other matter which could be handled best by co-operative action. 

\index{Breed purity|(}
The typical history of the formation of the British breeds\index{Origin of breeds} (the breeding practices in the United States are patterned 
closely after the British ones) was about as follows: First, came the existence of a type which was more useful and 
desirable than the ordinary type, but which was not yet distinctly different in pedigree from the other animals in the 
community. Second, some of the best animals of that type were gathered into one or a few herds which then ceased to 
introduce much outside blood. Then followed some rather intense \index{Inbreeding}inbreeding among these animals and their descendants until 
the animals of those herds became distinct from the other animals in the community, not only in type but also in 
inheritance; that is, until they were really welded into a breed. Third, if this process had been moderately successful in 
producing a desirable kind of animal, the breed became more and more popular and more and more herds were established. 
Fourth, necessity for a central herdbook arose when the breed became so numerous and the breeders so many that no man could 
remember all the information needed for the proper use of pedigrees. Fifth, a breed society was formed to safeguard the 
purity of the breed, conduct the herdbook, and promote the general interests of the breeders. From the very beginning many 
of these breeders emphasized that the males they produced were especially valuable for crossing on other races or on common 
stock. An important function of these \index{Breed purity}pure breeds was to produce sires for commercial use on unrelated stock, even for 
crossbreeding. 

In not all breeds did the breed history develop in just exactly these steps. Sometimes there was a breed society before 
there was a herdbook. Thus, even in Bakewell's time a Dishley Society was founded, with the primary object of protecting 
the pure breeding of the animals descended from those bred by Bakewell and the commercial promotion of the interests of 
those who were breeding animals of the Bakewell strains. Often there was no intervening stage of private ownership of the 
herdbook, but the breed society established the first herdbook itself. In practically every case the breed was a well-
established fact before any herdbook was considered. People did not say to each other: ``Let's establish a breed.'' Rather 
they said: ``Here we already have a useful and profitable breed. We should protect its purity and our own interests as 
possessors of this valuable breeding stock and the interests of the purchasers who want genuine animals of this breed.''\index{Breed purity|)}

In the continental countries of Europe pure breeding and registration were generally organized at a later date than in 
Britain. In Germany and adjoining lands (Engeler, 1936), extensive efforts at improvement developed first in sheep breeding,
\footnote{As long ago as 1779 Daubenton was measuring wool fineness with a micrometer and in 1802 Abilgaard wrote in detail 
about the reasons for marking sheep individually so that their production could be recorded and used as a basis for 
selections.} then in horse breeding, and then in cattle breeding. In the period about 1800 it was common practice to cross 
extensively, even for producing seedstock, in accordance with the idea expressed by Buffon (1780) that perfection could be 
attained only through widespread crossing and mixing of all individuals which had any of the desired points, regardless of 
race or regional origin. Then for a half century the trend changed toward following the successful English example of pure 
breeding, that is, of improving a breed from within itself.\footnote{Thus Kr\"{u}nitz wrote in 1815 with surprise: ``The 
English improve a race from within itself. They choose carefully the best individuals they can find within the same race 
and mate these together. In this way they keep the stock unmixed and produce a race in which the desired qualities are 
retained permanently.''} Some of the writers (but perhaps not many of the breeders themselves?) carried this to an extreme 
form in what became known as the theory of ``racial constancy.''\index{``Racial constancy'' dogma} This held that each animal transmitted according to its 
race and not according to its own characteristics. The latter were unimportant except as they indicated the animal's purity 
of race. Under the influence of that doctrine, herdbooks were only records of genealogy, and official attention was focused 
almost wholly on purity of breeding.

Sharp reaction to the theory of ``racial constancy'' developed about 1860, and the pendulum swung far the other way, at 
least among the writers. Thenceforth attention was devoted more to the individual. They sought more and more to make the 
herdbooks contain full information about each animal's characteristics, productivity, conformation, reproductive 
performance, longevity, etc. To collect this information the herdbook societies were organized around semi-official local 
records which might be either the private herdbook which the breeder was required to keep himself or the records kept by a 
local breeding association organized somewhat like a dairy herd improvement association in the United States. In either 
case the records to be kept were definitely prescribed and were inspected more or less regularly by officials of the 
herdbook society or of the government. From those local records the central herdbook society collected regularly the 
information thought useful there. 

Because of this background, the continental breed associations make more use of formal scoring or other inspections or 
production requirements as a prerequisite to registry than is done in Britain, where the responsibility of deciding whether 
a purebred animal is good enough for registry is still left almost entirely with the individual breeder. In Britain it is 
thought that the reputation of his herd and the resultant prices which the customers will pay will more or less automatically 
reward or penalize the breeder if his efforts have been above or below average. 

Often the continental associations have only tentative registry at birth; final registry in a printed herdbook is postponed 
until an animal is mature or even until it is dead and all of the data on its lifetime performance, prizes won, scores for 
type, etc., can be printed, too.

At the Strickhof agricultural school at Zurich, Switzerland, the production of all cows has been recorded continuously 
since 1871. The first cow-testing association in the world was established in 1892 at Vejen in Denmark. For the last two 
decades about 40 per cent of the cows in Denmark have had their milk weighed and tested. Those thought to be best among 
these are admitted to registry each year. Often the continental associations were built around some such plan for recording 
production. At the beginning of 1938, 67 per cent of all cows in Germany were on test but the figure had only very recently 
risen that high. 

\index{Breed purity|(}
\index{Grading}In the lands of their origin the breeds usually continued for a long time to register what would be called high grades\index{Registration of grades} in 
the United States. A common rule --- which still holds there for many breeds --- was that females with four top-crosses of 
registered sires were eligible to registry themselves if they came up to certain standards of individual excellence. In 
importing lands, such as the United States and Argentina, the herdbooks have usually been closed from the very beginning, 
and fashions in pedigrees have often gone to greater extremes in waves of speculation than has been the case in the native 
lands of the breeds. The greater emphasis which the importing countries placed on strict purity of breeding is illustrated 
by the fact that for some breeds, e.g., Berkshire swine, Holstein-Friesian and Ayrshire cattle, and Hampshire sheep, 
herdbooks were established in the United States before they were in the native land of the breed.

Breeders of poultry have never attempted general registration of all eligible individuals. The short life and comparatively 
small value of the individual birds have made that uneconomical. There have, however, been a number of attempts to register 
individuals in connection with a scheme of advanced registry for outstanding producers, e.g., Lancashire (England) Poultry 
Society, Record of Performance in Canada, the Record of Performance in the United States.

As an illustration of the difficulties encountered in assembling the first herdbooks, we may take the pedigree of the first bull in the Coates herdbook and wonder how Mr. Coates collected and verified the infor.mation printed for it. That pedigree is as follows: 

\begin{quote}
"(I) Abelard, Calved in 1812, bred by Major Bower; got by Cecil (120), d. (Easby) by Mr. Booth's Lame Bull (359), g.d. by Mr. Booth's Old White Bull (89), gr. g.d. bought at Darlington."
\end{quote}

\noindent
For a more specific account of some of the kinds of mistakes later found in those first herdbooks, consult the second 
edition of ``The Polled Herd Book'' (Aberdeen-Angus in Scotland) and read the preface and the notes in brackets under the 
pedigrees of bulls numbers 1, 2, 3, 4, 12, 17, 29, 35, 49 and 51. 

As an example of the controversies which arose over the \index{Breed purity}purity or non-purity of certain animals may be cited the long 
controversy in early American Shorthorn history over the ``seventeens''\footnote{See pp. 165--72 of ``Shorthorn Cattle,'' by 
Alvin H. Sanders. Sanden Publishing Company. 1918.} which were imported in 1817 and hence were not recorded in the Coates 
Herdbook since it did not appear until five years later. Also, Bates makes many references to the long controversy over the 
``Galloway alloy,'' which one of the early Shorthorn breeders was thought by some of his fellows to have introduced into 
his Shorthorns. 
\index{Breed associations|)}
\index{Breed purity|)}
\index{Herdbooks|)}

\section*{REFERENCES}

\begin{hangparas}{0.5in}{1}%
Anonymous. 1894. Jour. Royal Agr. Soc. of England, 55:1-31.

Allen, R. L. 1847. Domestic animals. 227 pp. New York: Orange Judd \& Company. American Shorthorn Herdbook, 2:33--69.

``The Major.'' 1920. Robert Blakewell's great work. The Shorthorn World and Farm Magazine, 5 (No. 7):3--4 and 63--65.

Bates, Thomas. 1871. History of improved Shorthorn cattle (from the notes of Thomas Bates, edited by his foreman, Thomas 
Bell). See especially pp. 11, 19, 37--38, 213--15 and 226.

Engeler, W. 1936. (In) Neue Forschungen in Tierzucht. Bern. Pp. 39--70.

Harrison, Fairfax. 1917. Roman farm management. New York: The Macmillan Company.

Sanders, A.H. 1915. At the sign of the Stockyard Inn. (Especially pp. 45--88.) Chicago: Sanders Publishing Company.
---. 1918. Shorthorn cattle. (Especially pp. 29--113.) Chicago: Sanders Publishing Company.

---. 1936. Red, White and Roan. (See especially pp. 1--24.) Chicago: American Shorthorn Breeders' Association.

Van Riper, Walker. 1932. Aesthetic notions in animal breeding. Quart. Rev. of Biology, 7:84--92.

Wilson, James. 1926. The history of stockbreeding and the formation of breeds. In Proc. of the Scottish Cattle Breeders Conference for 1925, pp. 17--25.
\end{hangparas}
\index{Origin of breeds|)}
\index{Pedigree breeding, beginnings of|)}