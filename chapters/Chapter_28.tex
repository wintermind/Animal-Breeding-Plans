\chapter{Mating Unlike Individuals}
\label{cha:Lush_Chapter_28}
\index{Heterozygosis|(}
\index{Homozygosis|(}
\index{Mating unlikes|(}

The mating of unlike individuals (negative assortive mating on the
basis of somatic resemblance) is most commonly practiced either to
mark time while the breeder is deciding what his goal is to be or, where
the ideal is an intermediate, to correct defects by mating each animal
to one which is equally extreme but in the opposite direction. This is
sometimes called \textit{compensatory} mating.

Everyone does some of this from time to time, at least for minor
traits. There are no absolutely perfect animals. A breeder usually realizes
that his females are good in some respects but below his standard in
others. Under such circumstances he is almost certain to seek for his
next sire one which is particularly strong where his females are weak.
Since he cannot find a sire which is absolutely perfect in all respects, he
will accept one which is a little below his standard in characteristics for
which the females are unusually good. In the breeding of Rambouillet
or Merino sheep, it is common practice for breeders to seek ``light C''
rams to mate to their ``heavy B'' ewes and ``heavy B'' rams to use on
their ``light C'' ewes. Many are confident that this type of mating is
more apt to produce a high percentage of lambs which are on the borderline
of being the desired heavy C's or light B's than would be produced
from parents both of which were the desired type. Another
prominent example of mating unlikes is in the breeding of dual-purpose
cattle, where there is considerable mating of those which vary most
toward the extreme dairy type with others which vary most toward the
extreme beef type.

\section*{CONSEQUENCES}
\index{Genes, number of|(}

The consequences of mating unlikes are the reverse of those of mating
like to like. Heterozygosis is increased only a very little. The maximum
effect of mating unlikes, even if continued indefinitely, would be
to make the heterozygos1s $\dfrac{2n(1 - m)}{2n(1 - m) + m}$ of what it
would be under random mating, \textit{m} being negative. In the impossibly
extreme case when $m = -1.0$, that would increase the heterozygosis by only
$\dfrac{1}{4n - 1}$ of what it would be under random mating. That would
increase heterozygosis by one-third if only one pair of genes were involved,
by one-seventh if there were two pairs, only by one-eleventh if there were
three, etc. If the value of \textit{m} is nearer zero, the power of this
breeding system to affect heterozygosis will be still further reduced. When
\textit{m} is as near zero as $-.2$, the increase in heterozygosis cannot
exceed one-eleventh of the original amount even when \textit{n} is 1 and
cannot exceed one forty-seventh if \textit{n} is 4. Since \textit{m} must
usually be low and \textit{n} may well be large, it is obvious that average
homozygosis is scarcely affected at all.

The mating of unlikes\index{Variation!decreased by mating unlikes|(} together makes the correlation between parent
and offspring distinctly lower, since the two parents are quite different
from each other and the effects of the genes which an offspring
inherits from one tend to be canceled by the effects of the genes it inherits
from the other. Likewise, this system of mating tends to lower the
conelation between brothers, although not as much as the correlation
between parent and offspring is lowered.

Mating unlikes together tends to make the whole population uniform since an
extreme individual in one direction tends to be mated with one which is extreme
in the other. The offspring of each mating thereby usually average nearer the
population average than they would if mating were random. If mating were random,
there would be some matings where both parents happened to be extreme in the
same direction. This reduction in variability nearly reaches its limit within
the first and second generations after the mating of unlikes begins. In the very
first generation produced by mating unlike individuals, the variance becomes
$\frac{2 + m}{m}$ times what it was under random mating. Of course \textit{m}
is negative, but it has to be small unless heritability is high. The maximum
effect on the variability of a whole population occurs when \textit{n} is large
and \textit{m} is strongly negative. That maximum effect is to halve the variance
when $m = -1.0$ and \textit{n} is very large. If $m = -.4$, the maximum effect
is to reduce the variance two-sevenths when \textit{n} is very large and
one-sixth when \textit{n} is only 1. The original variability will reappear
almost at once when the mating of unlikes is abandoned.
\index{Genes, number of|)}

A system of mating unlikes is most useful when the desired type
is an intermediate. Under such conditions the maximum proportion of
desired individuals among the offspring can be obtained by mating
males which are of the desired type to females which are also of the
desired type and mating any breeding animals which deviate from the
ideal in one direction to those which deviate equally far in the other
direction.
\index{Heterozygosis|)}
\index{Homozygosis|)}

Mating unlikes together is extensively practiced commercially to
correct defects. This is a sound practice wherever there are enough
females undesirably extreme in one direction to justify the keeping of a
male equally extreme in the other direction.
\index{Variation!decreased by mating unlikes|)}

\section*{SUMMARY}

Mating unlikes together in the absence of selection leads to:
\begin{enumerate}
\item A more uniform population than under random mating, with a
larger percentage of intermediate offspring and fewer extremes in either
direction. This increased uniformity reaches nearly its full extent in
the first generation after the mating of unlikes is begun. If the mating of
unlikes ceases, the population returns almost at once to its original
variability under random mating.
\item Only a slight increase in heterozygosity under the very simplest
genetic situations and practically no change in heterozygosity under the
situations apt to be encountered.
\item A distinctly lower resemblance between parent and offspring and
a somewhat lower resemblance between other relatives than under random
mating.
\end{enumerate}
Mating unlikes is useful in holding the population together as a
more uniform group until the average type or some other intermediate
type can be fixed by close inbreeding.

Mating unlikes is useful in correcting defects wherever the ideal is
intermediate and some animals are too extreme in one direction while
others are too extreme in the other direction.

\section*{REFERENCES}

\begin{hangparas}{0.5in}{1}%
Wright, Sewall. 1921. Systems of mating. {III}. Assortive mating based on somatic
resemblance. Genetics, 6:144--6l. V. General considerations. Genetics, 6:167--78.
\end{hangparas}
\index{Assortive mating|)}
\index{Mating unlikes|)}